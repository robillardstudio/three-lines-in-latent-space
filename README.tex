% Options for packages loaded elsewhere
\PassOptionsToPackage{unicode}{hyperref}
\PassOptionsToPackage{hyphens}{url}
%
\documentclass[
]{article}
\usepackage{amsmath,amssymb}
\usepackage{lmodern}
\usepackage{ifxetex,ifluatex}
\ifnum 0\ifxetex 1\fi\ifluatex 1\fi=0 % if pdftex
  \usepackage[T1]{fontenc}
  \usepackage[utf8]{inputenc}
  \usepackage{textcomp} % provide euro and other symbols
\else % if luatex or xetex
  \usepackage{unicode-math}
  \defaultfontfeatures{Scale=MatchLowercase}
  \defaultfontfeatures[\rmfamily]{Ligatures=TeX,Scale=1}
\fi
% Use upquote if available, for straight quotes in verbatim environments
\IfFileExists{upquote.sty}{\usepackage{upquote}}{}
\IfFileExists{microtype.sty}{% use microtype if available
  \usepackage[]{microtype}
  \UseMicrotypeSet[protrusion]{basicmath} % disable protrusion for tt fonts
}{}
\makeatletter
\@ifundefined{KOMAClassName}{% if non-KOMA class
  \IfFileExists{parskip.sty}{%
    \usepackage{parskip}
  }{% else
    \setlength{\parindent}{0pt}
    \setlength{\parskip}{6pt plus 2pt minus 1pt}}
}{% if KOMA class
  \KOMAoptions{parskip=half}}
\makeatother
\usepackage{xcolor}
\IfFileExists{xurl.sty}{\usepackage{xurl}}{} % add URL line breaks if available
\IfFileExists{bookmark.sty}{\usepackage{bookmark}}{\usepackage{hyperref}}
\hypersetup{
  hidelinks,
  pdfcreator={LaTeX via pandoc}}
\urlstyle{same} % disable monospaced font for URLs
\usepackage{graphicx}
\makeatletter
\def\maxwidth{\ifdim\Gin@nat@width>\linewidth\linewidth\else\Gin@nat@width\fi}
\def\maxheight{\ifdim\Gin@nat@height>\textheight\textheight\else\Gin@nat@height\fi}
\makeatother
% Scale images if necessary, so that they will not overflow the page
% margins by default, and it is still possible to overwrite the defaults
% using explicit options in \includegraphics[width, height, ...]{}
\setkeys{Gin}{width=\maxwidth,height=\maxheight,keepaspectratio}
% Set default figure placement to htbp
\makeatletter
\def\fps@figure{htbp}
\makeatother
\setlength{\emergencystretch}{3em} % prevent overfull lines
\providecommand{\tightlist}{%
  \setlength{\itemsep}{0pt}\setlength{\parskip}{0pt}}
\setcounter{secnumdepth}{-\maxdimen} % remove section numbering
\ifluatex
  \usepackage{selnolig}  % disable illegal ligatures
\fi

\author{}
\date{}

\begin{document}

\begin{center}\rule{0.5\linewidth}{0.5pt}\end{center}

\hypertarget{des-moduxe8les-et-des-mots}{%
\section{Des modèles et des mots}\label{des-moduxe8les-et-des-mots}}

\textbf{\emph{Des Modèles et des Mots} est une série de publications dédiée à un ensemble de conversations sur le Computer Art et l'intelligence artificielle éditées par Gaëtan Robillard. La série entend présenter des entretiens avec Margit Rosen, Frieder Nake, Jérôme Nika, Véra Molnar, et Kazushi Mukaiyama. Les textes s'accompagnent d'une série d'images et de la diffusion d'un modèle d'apprentissage profond.}

Édition ESIPE -- formation IMAC (Université Gustave Eiffel).\\
Fr \textbar{} \href{https://github.com/robillardstudio/three-lines-in-latent-space/blob/main/README_EN.md}{En}

\begin{center}\rule{0.5\linewidth}{0.5pt}\end{center}

\hypertarget{trois-lignes-dans-un-espace-latent}{%
\subsection{Trois lignes dans un espace latent}\label{trois-lignes-dans-un-espace-latent}}

Modèle : Gaëtan Robillard et Wendy Gervais.

\hypertarget{avant-propos}{%
\subsubsection{Avant-propos}\label{avant-propos}}

Le code présenté ici soutient un travail exploratoire sur les GANs (Generative Adversarial Networks, ou Réseaux antagonistes génératifs)\footnote{Le modèle de GAN présenté ici a été adapté à partir de l'ouvrage \emph{Generative Deep Learning} de David Foster (O'Reilly).}. Ces modèles d'architecture sont établis depuis quelques années dans la recherche en informatique, employés pour produire des médias de synthèse à partir de larges bases de données d'images. Ce répertoire github est pensé en tant qu'environnement de travail ou \emph{framework}, base à compléter et à explorer, pour quiconque souhaiterait découvrir, façonner ou critiquer ce type de modèle dans un contexte de recherche et d'expérimentation visuelle.

Le plus souvent, les GANs sont entraînés à partir de grandes quantités de photographies ou de dessins produits à la main -- des données observées dans le réel. Les images sont figuratives. La philosophie du deep learning génératif semble s'accompagne d'un naturalisme qu'il faut remettre en cause. Qu'en est-il de l'abstraction géométrique ?\footnote{En faisant référence ici au \emph{Projet Mondrian} de Frieder Nake : « I had scanned all of the neoplastic images of Mondrian's, and had arranged them into a time sequence of Mondrian's painting. The idea then was to analyze them year by year, in all sorts of statistical directions. Out of this, I wanted to develop a predictory system that would predict and create the New York Boogie-Woogie! Outrageously optimistic! » Frieder Nake, « The Art of Being Precise: Frieder Nake in Conversation {[}With Margit Rosen{]} », ZKM, 2022. \url{https://www.youtube.com/watch?v=Z_pOiHX6HYE}}? Et qu'en est-il des images déjà produites par un code ? Nous nous intéressons ici à des données d'entraînement qui sont des données de synthèse -- des données générées par un algorithme et par le truchement de variables aléatoires -- en référence au champ du Computer Art\footnote{Le Computer Art est entendu ici comme un champ iconographique à part entière, caractérisable par de nombreuses références à l'abstraction géométrique, à l'art cinétique et à l'art conceptuel. Bien qu'il soit difficile de circonscrire ce champ, on peut caractériser le Computer Art par son inscription dans l'esthétique informationnelle et générative issue de théoriciens comme Max Bense et Abraham Moles. Sans s'y réduire totalement, l'emploi d'algorithmes générateurs de nombres pseudoaléatoires est un trait marquant de l'époque pionnière.}. À cette démarche s'ajoute une recherche sur l'image en mouvement, telle qu'elle apparaît en potentiel dans l'exploration de l'espace latent\footnote{L'espace latent est l'espace théorique contenant les ``points'', ou vecteurs, support des motifs récurrents interprétables par un réseau de type apprentissage profond génératif. Ces vecteurs ou inputs sont traités par le GAN afin de générer les images finales ou outputs.} d'un GAN. L'une des perspectives du modèle est bien la création de séquences d'images nouvelles, issues de fonctions de parcours dans l'espace ou la « vision » du modèle entraîné.

Principalement écrit en Python, décomposé en trois Notebooks distincts : Training, Inference, Inference+ -- le code est commenté de façon à guider le profane à travers les différentes parties du \emph{framework}. Les trois Notebooks présentés plus bas ont été configurés sur Google Colaboratory (aucune installation requise), et sont également exploitables dans un environnement local sous Jupyter\footnote{Par exemple dans un environnement comme Anaconda. Pour l'installation des dépendances, Cf. \href{https://github.com/leogenot/DeepDrawing}{Generative Deep Drawing}}.

D'une façon générale, l'apprentissage profond est un sujet technique complexe -- notamment du fait de la très grande dimension des réseaux de neurones (le terme de boîte noire est adapté pour désigner ce problème). Tout en réfléchissant à la lisibilité de ce type de modèle, c'est ici la recherche artistique et pédagogique qui doit être mise en avant. Autant que possible, les aspects visuels de la démarche sont présentés : les données d'entraînement, l'architecture du GAN et les résultats.

\begin{center}\rule{0.5\linewidth}{0.5pt}\end{center}

\hypertarget{donnuxe9es-dentrauxeenement}{%
\subsubsection{Données d'entraînement}\label{donnuxe9es-dentrauxeenement}}

« Trois lignes dans un espace latent » repose sur un ensemble de 10 000 images élémentaires et semblables, appelé « dataset », généré à l'aide d'un code Java écrit et exécuté dans l'environnement de programmation Processing. Le code permettant de générer ces images est disponible dans le \href{https://github.com/kaugrv/models_words/blob/main/lines/lines.pde}{dossier « lines »}.

Cet ensemble d'images constitue une seule et même classe. Chacune d'elles suit les mêmes règles de construction -- elles sont détaillées ci-après, et est strictement dictée par l'exécution du code. Cela en fait un dataset synthétique et algorithmique, comportant des images simples, chacune différente, mais appartenant à un ensemble. Cette homogénéité relative sera utile pour entraîner notre réseau.

Chaque image de notre dataset est au format 128 par 128 pixels. Sur fond noir, on génère 3 lignes d'épaisseur 5 pixels. On distingue une ligne verticale et deux horizontales. Leur position dans l'image est définie de façon aléatoire (loi uniforme ou loi gaussienne, selon). Leurs couleurs sont également aléatoires (loi uniforme sur {[}0,3{]}). Chacune des 3 lignes peut prendre l'une des couleurs suivantes \footnote{Les trois premières couleurs sont les couleurs primaires. Leurs valeurs numériques sont tirées du travail de Piet Mondrian (Cf. \emph{Trafalgar Square}, Piet Mondrian, 1943 ; reproduction in \emph{\href{https://www.moma.org/collection/works/79879}{Museum of Modern Art}}).} :

\begin{itemize}
\tightlist
\item
  blanc
\item
  jaune
\item
  rouge
\item
  bleu
\end{itemize}

\begin{figure}
\centering
\includegraphics{https://user-images.githubusercontent.com/103901906/179547588-a322ef87-3ee4-4d77-b573-b6f613bf541a.png}
\caption{Extrait du dataset}
\end{figure}

Par calcul, on peut approcher le nombre d'images contenues dans la classe. Chacune des 3 lignes comporte une abscisse (si verticale) ou une ordonnée (si horizontale) comprise entre 0 et 128, et possède une couleur choisie aléatoirement parmi 4. Le nombre de lignes verticales et horizontales étant fixé, on a donc :

\[ \prod_{i=1}^{3} (4 \times 128) = (4 \times 128)^3 = 134217728 \]

c'est-à-dire 134 millions d'images possibles. En générant 10 000 d'entre elles, on couvre 0.007\% d'images de la classe -- autant dire que nous sommes certains de générer 10 000 images uniques. Ce dataset sera chargé lors de l'entraînement, à l'aide de le la fonction \texttt{load\_dataset} de \href{https://github.com/kaugrv/models_words/blob/main/utils/loaders.py}{loaders.py}.

\hypertarget{architecture}{%
\subsubsection{Architecture}\label{architecture}}

Le réseau \emph{Trois lignes} est un GAN de type WGAN-GP, codé en Python à l'aide des librairies \emph{TensorFlow} et \emph{Keras}.

Rappelons qu'un GAN se constitue d'un générateur et d'un discriminateur (que l'on appelle critique). Le GAN repose sur la concurrence de ces deux réseaux : le générateur crée des échantillons visuels et les soumet au critique, qui à son tour leur donne un score. Ce score détermine si l'image analysée est une image observée (issue des données d'entraînement) ou une image produite par le générateur). Le générateur alors progresse pour engendrer des images de plus en plus enclines à tromper l'appréciation du critique.

La classe de cette architecture est définie dans le code \href{https://github.com/kaugrv/models_words/blob/main/models/WGANGP.py}{WGANGP.py}. Lors de l'entraînement, le modèle est instancié pour travailler sur des images de 128 x 128 ; à l'entrée du générateur le vecteur (Z) est de dimension 100. Voici également les paramètres choisis pour les différentes \emph{layers} (critique, générateur) constitués principalement de convolutions. Les paramètres des matrices de convolutions sont donnés par \emph{filters, kernel} et \emph{strides} :

\begin{verbatim}
gan = WGANGP(input_dim=(IMAGE_SIZE, IMAGE_SIZE, 3), 
             critic_conv_filters=[128, 256, 512, 1024],
             critic_conv_kernel_size=[5,5,5,5], 
             critic_conv_strides=[2, 2, 2, 2], 
             critic_batch_norm_momentum=None, 
             critic_activation='leaky_relu', 
             critic_dropout_rate=None, 
             critic_learning_rate=0.0002, 
             generator_initial_dense_layer_size=(8, 8, 512), 
             generator_upsample=[1, 1, 1, 1], 
             generator_conv_filters=[512, 256, 128, 3], 
             generator_conv_kernel_size=[10,10,10,10], 
             generator_conv_strides=[2, 2, 2, 2], 
             generator_batch_norm_momentum=0.9, 
             generator_activation='leaky_relu', 
             generator_dropout_rate=None, 
             generator_learning_rate=0.0002, 
             optimiser='adam', 
             grad_weight=10, 
             z_dim=100, 
             batch_size=BATCH_SIZE
             )
\end{verbatim}

La matrice de convolution choisie pour le générateur est de taille 10 x 10 :

\texttt{generator\_conv\_kernel\_size={[}10,10,10,10{]}}

La convolution est ainsi relativement grande, de façon à ce que le générateur réponde à la qualité abstraite des images en considérant à une échelle globale. En revanche, le critique est associé à une matrice de convolution de taille inférieure, en taille 5 x 5 :

\texttt{critic\_conv\_kernel\_size={[}5,5,5,5{]}}

Les images sont scorées par le critique via une convolution de niveau plus locale que celle du générateur. En somme, si le générateur tend à l'abstraction visuelle -- la ligne, le critique porte son analyse dans les détails. Cette asymétrie macro-micro des matrices de convolution entre générateur et critique résulte d'une interrogation forte quant à la capacité du réseau en matière d'abstraction visuelle.

D'autres dimensions de la convolution pourraient être explorées en fonction de nouveaux datasets. Pour plus de détails sur la convolution, voir \emph{\href{https://github.com/vdumoulin/conv_arithmetic}{Convolution arithmetic}}.

\begin{figure}
\centering
\includegraphics{https://user-images.githubusercontent.com/103901906/179565557-e17c1acc-e9a2-48c1-aa5e-973d7caf03d2.png}
\caption{Générateur}
\end{figure}

\begin{figure}
\centering
\includegraphics{https://user-images.githubusercontent.com/103901906/179565848-3a8334e6-1680-4085-9bad-cbcb81c8d142.png}
\caption{Critique}
\end{figure}

Ci-dessus : Représentations graphiques des couches constituant le générateur et le critique. Dans cette perspective, on observe l'articulation des couches et leurs dimensions successives, en condensé (au centre) et en proportions réelles (en haut à droite). Rendu réalisé avec \href{https://github.com/alexlenail/NN-SVG}{NN-SVG}.

\hypertarget{entrauxeenement}{%
\subsubsection{Entraînement}\label{entrauxeenement}}

\begin{center}\rule{0.5\linewidth}{0.5pt}\end{center}

Notebook \href{https://colab.research.google.com/drive/12WCzKlR--V8E7HMZHJ89nobVDCknCKmE\#scrollTo=C7vmECVpwSZm}{Training} sur Google Colab.

\begin{center}\rule{0.5\linewidth}{0.5pt}\end{center}

Après avoir cloné ce dépôt, on travaillera à sa racine : \texttt{/models\_words}. Il est possible d'utiliser le dataset \emph{lines} décrit plus haut ou bien d'en utiliser un nouveau (en le chargeant directement via une cellule dans l'environnement Colab ou en remplaçant le fichier \texttt{data.zip}). Au moins 10 000 images (format 128x128) sont conseillées. Selon la complexité des images, davantage de données peuvent être nécessaires.

Les premières cellules du Notebook servent à charger le modèle et les bibliothèques, puis à charger le dataset. Il est possible de configurer les paramètres de l'entraînement, notamment le numéro associé à l'entraînement (\texttt{RUN\_ID}), ou le \texttt{BATCH\_SIZE}, c'est-à-dire le nombre d'images du dataset présentées au réseau lors d'une itération de l'entraînement. Par défaut, le \texttt{BATCH\_SIZE} est paramétré à 16. Augmenter cette taille rendra le temps de calcul plus long, mais permettra d'améliorer la précision du réseau (garder une puissance de 2, comme 32, 64 ou 128).

D'autres paramètres de l'entraînement sont modifiables :

\begin{itemize}
\tightlist
\item
  \texttt{EPOCHS} : le nombre de cycles d'expositions du réseau au dataset complet (un cycle est composé de n batchs). Pour ce modèle, le paramètre par défaut est de 1201 époques, mais selon la complexité des images à analyser, il peut être inférieur ou supérieur (jusqu'à 60 000 époques et plus).
\item
  \texttt{PRINT\_EVERY\_N\_BATCHES} : au lieu de rendre et d'enregistrer une image ou output à chaque époque, on peut choisir une autre fréquence. Les images générées se trouveront dans le dossier \texttt{run/gan}, dans un sous dossier correspondant au numéro de l'entraînement. Par exemple, pour \texttt{EPOCHS\ =\ 1201} et \texttt{PRINT\_EVERY\_N\_BATCHES\ =\ 100}) on obtiendra :
\end{itemize}

\begin{figure}
\centering
\includegraphics{https://user-images.githubusercontent.com/103901906/176936405-c4fbce75-1ece-419f-a47e-5bd0e4547ef4.png}
\caption{Exemple samples}
\end{figure}

\begin{itemize}
\tightlist
\item
  \texttt{rows} et \texttt{columns} : au cours de l'entraînement le rendu est paramétré pour présenter une grille d'images générées par le réseau. Ces deux variables permettent de paramétrer la densité de la grille. Voici un exemple avec les valeurs \texttt{rows\ =\ 5}, \texttt{columns\ =\ 5} et \texttt{rows\ =\ 2}, \texttt{columns\ =\ 2} :
\end{itemize}

\includegraphics{https://user-images.githubusercontent.com/103901906/176937177-ef705ff3-603f-4fb1-9243-b15b1783b3a0.png} \includegraphics{https://user-images.githubusercontent.com/103901906/176937301-ef2df143-63bb-4dad-8ce9-86d777c364f6.png}

En sortie de l'entraînement, il est possible d'observer l'évolution de fonction de perte (\emph{Loss}) du critique D et du générateur :

\begin{verbatim}
...
452 (5, 1) [D loss: (-92.2)(R -71.9, F -58.8, G 3.8)] [G loss: 41.4]
453 (5, 1) [D loss: (-116.2)(R -179.5, F 4.4, G 5.9)] [G loss: 1.4]
454 (5, 1) [D loss: (-95.4)(R -113.7, F -34.5, G 5.3)] [G loss: 46.7]
455 (5, 1) [D loss: (-104.6)(R -114.9, F -40.9, G 5.1)] [G loss: 51.2]
456 (5, 1) [D loss: (-108.6)(R -120.9, F -28.5, G 4.1)] [G loss: 5.9]
457 (5, 1) [D loss: (-82.6)(R -87.7, F -38.1, G 4.3)] [G loss: 40.1]
458 (5, 1) [D loss: (-92.0)(R -173.4, F 16.8, G 6.5)] [G loss: -37.0]
459 (5, 1) [D loss: (-120.8)(R -118.1, F -61.4, G 5.9)] [G loss: 61.5]
460 (5, 1) [D loss: (-101.2)(R -94.1, F -60.1, G 5.3)] [G loss: 84.7]
461 (5, 1) [D loss: (-86.1)(R -167.7, F 22.9, G 5.9)] [G loss: -37.0]
462 (5, 1) [D loss: (-92.4)(R -127.9, F -16.1, G 5.2)] [G loss: 26.5]
463 (5, 1) [D loss: (-109.2)(R -116.9, F -36.9, G 4.5)] [G loss: 7.6]
464 (5, 1) [D loss: (-103.0)(R -105.2, F -52.5, G 5.5)] [G loss: 30.6]
465 (5, 1) [D loss: (-76.8)(R -90.0, F -34.5, G 4.8)] [G loss: 58.8]
466 (5, 1) [D loss: (-107.0)(R -145.8, F -23.6, G 6.2)] [G loss: 13.6]
467 (5, 1) [D loss: (-106.2)(R -191.6, F 15.8, G 7.0)] [G loss: -42.8]
...
\end{verbatim}

La progression de ces valeurs à travers l'entraînement peut également être observée sous forme graphique :

On peut ainsi analyser la convergence des deux \emph{loss} : ici par exemple, la différence moyenne entre les images observées (\emph{real}, issues des données d'entraînement) et synthétiques (\emph{fake}, générées par le générateur) tend vers 0 à partir d'environ 1100 époques. À ce stade de l'entraînement, les images générées et les données d'entraînement sont évaluées à égalité par le critique. L'image de ce graphe est enregistrée dans \texttt{RUN\_FOLDER+"/images/Converge.png"}.

Dans le Notebook Colab, la cellule \textbf{Results} permet de télécharger l'ensemble des images issues de l'entraînement ; et le fichier generator.h5 -- le modèle entraîné du générateur, indispensable pour la partie \emph{Inference}. Notons qu'à la suite de l'entraînement, le critique n'a plus d'utilité \footnote{Aussi curieux que cela puisse paraître, le critique n'a en effet de rôle que lors de l'entraînement. Il serait intéressant d'imaginer un nouvel emploi pour le critique qui est alors délaissé. Une piste de réflexion : dans \emph{Artificial Aesthetics: a critical guide to media and design}, Lev Manovich et Emanuele Arielli proposent de considérer à égale importance la fonction analytique et générative d'un réseau de deep learning, la fonction analytique étant attribuée à une fonction d'évaluation esthétique des artefacts culturels.}.

\hypertarget{infuxe9rence}{%
\subsubsection{Inférence}\label{infuxe9rence}}

\begin{center}\rule{0.5\linewidth}{0.5pt}\end{center}

Notebook \href{https://colab.research.google.com/drive/13g3rX2zgyxT5YKTZILBrISybmLJ4_pXi}{Inference} sur Google Colab.

Pour utiliser le réseau entraîné, il faut se munir du fichier \emph{generator.h5}, voir partie précédente.

\begin{center}\rule{0.5\linewidth}{0.5pt}\end{center}

La partie \emph{Inference} se donne pour objectif de générer de nouvelles images par l'utilisation du modèle entraîné et par l'exploration de \emph{l'espace latent}. Cet espace est un espace vectoriel (ici en dimension 100) représentant l'ensemble des informations interprétables par le modèle. L'idée générale de l'exploration est d'étudier les images que le réseau est désormais capable de générer.

En utilisant le générateur entraîné, il va être possible de générer de nouvelles images relevant de la même catégorie que les images générées par le réseau lors de la dernière époque de son entraînement. Il sera possible de les étudier et de les \emph{interpoler} entre elles (c'est-à-dire d'introduire des vecteurs intermédiaires entre des vecteurs choisis).

La première des fonctions, \texttt{generate\_latent\_points}, génère un ou plusieurs vecteurs 100 (on choisit le nombre de vecteurs à générer en paramétrant \texttt{nb\_vec}). Chaque vecteur est déterminé par une fonction aléatoire qui détermine 100 valeurs flottantes aléatoires comprises entre environ -3 et 3 (la distribution répond à une loi normale centrée ou \emph{standard normal distribution}). Par défaut, on utilisera \texttt{latent\_dim\ =\ 100} et \texttt{nb\_vec}, et \texttt{nb\_vec\ =\ 2}. Ensuite nous étudions l'interpolation entre deux vecteurs de façon à générer une séquence d'images.

La seconde fonction \texttt{interpolate\_points} permet l'interpolation, c'est-à-dire de créer de nouveaux vecteurs situés entre les deux vecteurs créés dans la fonction précédente. Si par exemple on interpole deux vecteurs avec 20 autres (\texttt{nb\_img\ =\ 20}), il sera possible de créer alors une animation continue avec les 22 images correspondantes à chacun de ces vecteurs. Pour cela, la fonction \texttt{plot\_generated} génère, enregistre et affiche toutes ces images.

\begin{figure}
\centering
\includegraphics{https://user-images.githubusercontent.com/103901906/177205129-48acd30e-9a0b-4f2b-8450-ea58f21e3d83.gif}
\caption{Animation 1}
\end{figure}

En résumé, voici les trois étapes principales pour explorer et interpoler les points de l'espace latent et leurs images associées :

\begin{enumerate}
\def\labelenumi{\arabic{enumi}.}
\tightlist
\item
  avec la fonction \texttt{generate\_latent\_points}, on génère un tableau \texttt{pts} de format (2,100) qui contient 2 vecteurs 100 ; le nombre de vecteurs est choisi avec \texttt{nb\_vec}
\item
  avec la fonction \texttt{interpolate\_points}, on interpole les deux premiers tableaux de \texttt{pts}, ce qui donne le nouveau tableau \texttt{interpolated}
\item
  avec la fonction \texttt{plot\_generated}, on génère les images associées en passant le tableau \texttt{interpolated} en argument
\end{enumerate}

On peut ensuite télécharger l'ensemble de ces images au format zip, à l'aide de la dernière cellule de la partie \textbf{Function}. La partie \textbf{Super Resolution} sera présentée plus bas.

\hypertarget{infuxe9rence-1}{%
\subsubsection{Inférence +}\label{infuxe9rence-1}}

\begin{center}\rule{0.5\linewidth}{0.5pt}\end{center}

Notebook \href{https://colab.research.google.com/drive/14oww73GEQrECNtgaj8iK78jSw8GtHIiE?usp=sharing}{Inference +} sur Google Colab.

\begin{center}\rule{0.5\linewidth}{0.5pt}\end{center}

Pour aller plus loin, le Notebook Inference +, propose quelques fonctionnalités supplémentaires :

\begin{itemize}
\tightlist
\item
  visualiser le vecteur Z (vecteur 100 ici) sous la forme d'une image de 10x10 carrés gris, avec la fonction \texttt{printZ} ; voici quelques exemples d'images produites par le générateur, accompagnées de la visualisation du vecteur Z associée :
\end{itemize}

Attention, dans cette visualisation, l'échelle des couleurs est relative (un niveau de gris ne représentera pas la même valeur d'une visualisation à une autre).

\begin{itemize}
\tightlist
\item
  de là, étudier le vecteur Z et son influence sur les images produites en sortie, en laissant de côté la fonction initiale \texttt{generate\_latent\_point} (Cf. Inférence), mais en fabriquant un tableau (\texttt{pts}) avec d'autres méthodes ; les résultats d'une première étude en image fixe et en image animée sont disponibles dans le dossier \texttt{Z-anim}. Ici les valeurs ont été affichées pour donner une grille de proportion rectangulaire, accompagnées des valeurs numériques du vecteur (on remarque la relativité de l'échelle de gris) :
\end{itemize}

\begin{figure}
\centering
\includegraphics{https://user-images.githubusercontent.com/103901906/179784656-8ac40368-075b-42d3-b950-da614857c4dc.gif}
\caption{anim1}
\end{figure}

Deux autres fonctionnalités ont été ajoutées pour :

\begin{itemize}
\tightlist
\item
  générer des images en quantité, et par paires \emph{non interpolées} ; le nombre de paires est paramétrable avec \texttt{nb\_inf}
\item
  exporter dans un fichier texte les deux vecteurs Z associés à une paire de deux images ; il est alors possible d'importer les coordonnées d'un vecteur en tant qu'input du modèle, obtenir ainsi de nouveau l'image associée au vecteur donné
\end{itemize}

Ces deux fonctionnalités sont pensées pour de futures recherches.

\hypertarget{super-ruxe9solution}{%
\subsubsection{Super-résolution}\label{super-ruxe9solution}}

\begin{center}\rule{0.5\linewidth}{0.5pt}\end{center}

Notebooks \href{https://colab.research.google.com/drive/13g3rX2zgyxT5YKTZILBrISybmLJ4_pXi}{Inference} et \href{https://colab.research.google.com/drive/13g3rX2zgyxT5YKTZILBrISybmLJ4_pXi}{Inference +}

\begin{center}\rule{0.5\linewidth}{0.5pt}\end{center}

Pour augmenter la résolution des images interpolées, nous proposons d'utiliser un modèle de deep learning entraîné et distribué par Tensorflow \footnote{Voir sur Tfhub : \href{https://tfhub.dev/captain-pool/esrgan-tf2/1}{esrgan by captain-pool}}. Dans les deux notebooks en question, la partie \textbf{Super Resolution} s'applique directement sur les images obtenues en sortie du modèle.

On passe ainsi d'une image de 128x128 à une image de 512x512 (la définition de l'image est multipliée par 4).

\includegraphics{https://user-images.githubusercontent.com/103901906/177224935-3e7ec9c7-af83-490f-a78c-91088bfbff76.png} \includegraphics{https://user-images.githubusercontent.com/103901906/177224950-3936d167-81d9-44ca-9824-932b2aabdeb5.jpg}

L'algorithme s'applique sur toutes les images issues de l'interpolation. La dernière cellule permet de télécharger les super-résolutions, dans un fichier zip.

\begin{figure}
\centering
\includegraphics{https://user-images.githubusercontent.com/103901906/177619339-3cf28dfd-ff00-4761-a659-66a02d5e5abe.gif}
\caption{Animation Super-résolution}
\end{figure}

\hypertarget{ruxe9sultats}{%
\subsubsection{Résultats}\label{ruxe9sultats}}

Il est intéressant de comparer les images issues des données d'entraînement et les images générées par le modèle. Cette comparaison révèle des différences esthétiques importantes. Si les images d'origine contenaient des lignes nettes, unicolores, et toujours au nombre de 3 (deux horizontales et une verticale), les images générées augmentent ou diminuent ce nombre pour parfois faire apparaître jusqu'à 4 ou 5 lignes. Il y a une nouveauté géométrique et topologique. De plus, se dégage des résultats une impression de profondeur, due notamment à des contrastes clairs/obscurs, pourtant absents des données d'entraînement.

Certaines traces semblent plus lumineuses, évoquant des raies spectrales. Leur texture apparaît légèrement bruitée, une marque des processus de convolution. Ce grain est renforcé par la super-résolution. De nouvelles couleurs apparaissent (orange, bleu clair, beige) ainsi que des dégradés, mais tous les mélanges possibles ne sont pas visibles (le bleu et le jaune originaux auraient pu donner du vert, ce qui n'a jamais été observé dans nos entraînements). Ces « innovations esthétiques » sont inhérentes à la forme et aux propriétés du réseau. Pour autant elles préservent des qualités dont le rapprochement avec les images d'origine est évident (verticalité, horizontalité, couleurs primaires, épaisseurs moyennes \ldots).

\includegraphics{https://user-images.githubusercontent.com/103901906/177868869-e3375d52-a76c-4ecc-9f12-1cf25d79e345.png} \includegraphics{https://user-images.githubusercontent.com/103901906/177869223-8f200f12-21a9-4187-a36a-fd495b5e185f.png} \includegraphics{https://user-images.githubusercontent.com/103901906/177869230-faa33f57-f0f3-4f3a-b70b-552fc22f6b82.png} \includegraphics{https://user-images.githubusercontent.com/103901906/177869257-728d5fd4-2fc4-4b77-84ec-822620a7bbef.png} \includegraphics{https://user-images.githubusercontent.com/103901906/177869431-e225bcbe-8788-4d3b-bbb6-e09d68466b80.png} \includegraphics{https://user-images.githubusercontent.com/103901906/177869460-96cbca56-69e7-48df-a570-9a0904b2825e.png}

Le rapport entre les premières images, celle des données d'entraînement et les secondes -- les images générées par le modèle -- pose un certain nombre de questions sur la nature des images numériques, entre représentation et abstraction, entre originalité et reconstruction. L'espace latent du modèle de deep learning ouvre une voie pour formuler ces questions et explorer ce rapport par le travail de l'image en mouvement.

\hypertarget{ruxe9fuxe9rences}{%
\subsubsection{Références}\label{ruxe9fuxe9rences}}

David Foster, \emph{Generative Deep Learning: Teaching Machines to Paint, Write, Compose, and Play}, 2019. Code \href{https://github.com/davidADSP/GDL_code}{(https://github.com/davidADSP/GDL\_code)}

\end{document}
